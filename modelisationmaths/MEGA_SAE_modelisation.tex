% Options for packages loaded elsewhere
\PassOptionsToPackage{unicode}{hyperref}
\PassOptionsToPackage{hyphens}{url}
%
\documentclass[
]{article}
\usepackage{amsmath,amssymb}
\usepackage{iftex}
\ifPDFTeX
  \usepackage[T1]{fontenc}
  \usepackage[utf8]{inputenc}
  \usepackage{textcomp} % provide euro and other symbols
\else % if luatex or xetex
  \usepackage{unicode-math} % this also loads fontspec
  \defaultfontfeatures{Scale=MatchLowercase}
  \defaultfontfeatures[\rmfamily]{Ligatures=TeX,Scale=1}
\fi
\usepackage{lmodern}
\ifPDFTeX\else
  % xetex/luatex font selection
\fi
% Use upquote if available, for straight quotes in verbatim environments
\IfFileExists{upquote.sty}{\usepackage{upquote}}{}
\IfFileExists{microtype.sty}{% use microtype if available
  \usepackage[]{microtype}
  \UseMicrotypeSet[protrusion]{basicmath} % disable protrusion for tt fonts
}{}
\makeatletter
\@ifundefined{KOMAClassName}{% if non-KOMA class
  \IfFileExists{parskip.sty}{%
    \usepackage{parskip}
  }{% else
    \setlength{\parindent}{0pt}
    \setlength{\parskip}{6pt plus 2pt minus 1pt}}
}{% if KOMA class
  \KOMAoptions{parskip=half}}
\makeatother
\usepackage{xcolor}
\usepackage[margin=1in]{geometry}
\usepackage{color}
\usepackage{fancyvrb}
\newcommand{\VerbBar}{|}
\newcommand{\VERB}{\Verb[commandchars=\\\{\}]}
\DefineVerbatimEnvironment{Highlighting}{Verbatim}{commandchars=\\\{\}}
% Add ',fontsize=\small' for more characters per line
\usepackage{framed}
\definecolor{shadecolor}{RGB}{248,248,248}
\newenvironment{Shaded}{\begin{snugshade}}{\end{snugshade}}
\newcommand{\AlertTok}[1]{\textcolor[rgb]{0.94,0.16,0.16}{#1}}
\newcommand{\AnnotationTok}[1]{\textcolor[rgb]{0.56,0.35,0.01}{\textbf{\textit{#1}}}}
\newcommand{\AttributeTok}[1]{\textcolor[rgb]{0.13,0.29,0.53}{#1}}
\newcommand{\BaseNTok}[1]{\textcolor[rgb]{0.00,0.00,0.81}{#1}}
\newcommand{\BuiltInTok}[1]{#1}
\newcommand{\CharTok}[1]{\textcolor[rgb]{0.31,0.60,0.02}{#1}}
\newcommand{\CommentTok}[1]{\textcolor[rgb]{0.56,0.35,0.01}{\textit{#1}}}
\newcommand{\CommentVarTok}[1]{\textcolor[rgb]{0.56,0.35,0.01}{\textbf{\textit{#1}}}}
\newcommand{\ConstantTok}[1]{\textcolor[rgb]{0.56,0.35,0.01}{#1}}
\newcommand{\ControlFlowTok}[1]{\textcolor[rgb]{0.13,0.29,0.53}{\textbf{#1}}}
\newcommand{\DataTypeTok}[1]{\textcolor[rgb]{0.13,0.29,0.53}{#1}}
\newcommand{\DecValTok}[1]{\textcolor[rgb]{0.00,0.00,0.81}{#1}}
\newcommand{\DocumentationTok}[1]{\textcolor[rgb]{0.56,0.35,0.01}{\textbf{\textit{#1}}}}
\newcommand{\ErrorTok}[1]{\textcolor[rgb]{0.64,0.00,0.00}{\textbf{#1}}}
\newcommand{\ExtensionTok}[1]{#1}
\newcommand{\FloatTok}[1]{\textcolor[rgb]{0.00,0.00,0.81}{#1}}
\newcommand{\FunctionTok}[1]{\textcolor[rgb]{0.13,0.29,0.53}{\textbf{#1}}}
\newcommand{\ImportTok}[1]{#1}
\newcommand{\InformationTok}[1]{\textcolor[rgb]{0.56,0.35,0.01}{\textbf{\textit{#1}}}}
\newcommand{\KeywordTok}[1]{\textcolor[rgb]{0.13,0.29,0.53}{\textbf{#1}}}
\newcommand{\NormalTok}[1]{#1}
\newcommand{\OperatorTok}[1]{\textcolor[rgb]{0.81,0.36,0.00}{\textbf{#1}}}
\newcommand{\OtherTok}[1]{\textcolor[rgb]{0.56,0.35,0.01}{#1}}
\newcommand{\PreprocessorTok}[1]{\textcolor[rgb]{0.56,0.35,0.01}{\textit{#1}}}
\newcommand{\RegionMarkerTok}[1]{#1}
\newcommand{\SpecialCharTok}[1]{\textcolor[rgb]{0.81,0.36,0.00}{\textbf{#1}}}
\newcommand{\SpecialStringTok}[1]{\textcolor[rgb]{0.31,0.60,0.02}{#1}}
\newcommand{\StringTok}[1]{\textcolor[rgb]{0.31,0.60,0.02}{#1}}
\newcommand{\VariableTok}[1]{\textcolor[rgb]{0.00,0.00,0.00}{#1}}
\newcommand{\VerbatimStringTok}[1]{\textcolor[rgb]{0.31,0.60,0.02}{#1}}
\newcommand{\WarningTok}[1]{\textcolor[rgb]{0.56,0.35,0.01}{\textbf{\textit{#1}}}}
\usepackage{graphicx}
\makeatletter
\def\maxwidth{\ifdim\Gin@nat@width>\linewidth\linewidth\else\Gin@nat@width\fi}
\def\maxheight{\ifdim\Gin@nat@height>\textheight\textheight\else\Gin@nat@height\fi}
\makeatother
% Scale images if necessary, so that they will not overflow the page
% margins by default, and it is still possible to overwrite the defaults
% using explicit options in \includegraphics[width, height, ...]{}
\setkeys{Gin}{width=\maxwidth,height=\maxheight,keepaspectratio}
% Set default figure placement to htbp
\makeatletter
\def\fps@figure{htbp}
\makeatother
\setlength{\emergencystretch}{3em} % prevent overfull lines
\providecommand{\tightlist}{%
  \setlength{\itemsep}{0pt}\setlength{\parskip}{0pt}}
\setcounter{secnumdepth}{-\maxdimen} % remove section numbering
\ifLuaTeX
  \usepackage{selnolig}  % disable illegal ligatures
\fi
\usepackage{bookmark}
\IfFileExists{xurl.sty}{\usepackage{xurl}}{} % add URL line breaks if available
\urlstyle{same}
\hypersetup{
  pdftitle={Rendu Mega\_Sae Méthodes Numériques},
  pdfauthor={Adrien - Antonin - Vladimir - Tristan - Titouan (groupe B2)},
  hidelinks,
  pdfcreator={LaTeX via pandoc}}

\title{Rendu Mega\_Sae Méthodes Numériques}
\author{Adrien - Antonin - Vladimir - Tristan - Titouan (groupe B2)}
\date{13/05/24}

\begin{document}
\maketitle

Utiliser des libraries R selon votre problèeme par exemple : ompr, boot,
. . .

\section{Fonction linéaire sous contraintes
linéaires}\label{fonction-linuxe9aire-sous-contraintes-linuxe9aires}

\subsection{Introduction}\label{introduction}

coucou En rapport avec notre sujet de MEGA - SAE2.01234 : Conception et
développement d'un outil d'aide et de pilotage à destination
d'organisateurs de mariage (entreprise professionnel), nous allons
chercher à optimiser le nombre de mariages de type A et B à organiser
afin de maximiser le profit, tout en respectant certaines contraintes
sur les ressources disponibles.

Il s'agit d'un problème d'optimisation mathématique sous contraintes,
mettant en oeuvre plusieurs fonctions linéaire. En effet, la fonction à
maximiser (le profit total) ainsi que les contraintes définies (nombre
de jours, nombre de personnes, nombre de logements, nombre
d'intervenants) sont toutes des fonctions linéaires des variables de
décision xA et xB.

Pour répondre le plus possible à la consigne, ce document est divisé en
3 partie : - Partie modélisation : Définition de la fonction à optimiser
ainsi que les variables - Partie modélisation : Définition des
contraintes sur les variables - Partie Résolution : résolution de notre
problème(partie mathématiques + implémentation informatique (fonction
optim))

Enfin, nous commenterons et interpréterons la solution obtenue afin de
valider la pertinence de notre approche pour répondre à la problématique
initiale.

\subsection{Modélisation - Définition de la fonction à
optimiser}\label{moduxe9lisation---duxe9finition-de-la-fonction-uxe0-optimiser}

Imaginons que nous avons deux types de fêtes de mariage, A et B, avec
différents coûts et exigences en ressources. Nous cherchons à maximiser
le profit total en organisant un certain nombre de fêtes de type A et B.
Nous définissons les variables suivantes : - xA : nombre de fêtes de
type A à organiser - xB : nombre de fêtes de type B à organiser

Chaque mariage doit respecter les conditions suivantes :

Un mariage c'est environ 135€ par invité. Un mariage de type A coûterait
6750€, tandis qu'un mariage de type B coûterait environ 14000€.
L'objectif est d'optimiser le profit.

La fonction à maximiser serait donc la suivante : \[
6750\times X_a + 14000\times X_b
\]

\subsection{Modélisation - Définition des
contraintes}\label{moduxe9lisation---duxe9finition-des-contraintes}

\begin{itemize}
\tightlist
\item
  On note Xa et Xb le nombre de mariages de type A et B à organiser
  respectivement.
\item
  Notre objectif est de maximiser la fonction \[
  6750\times X_a + 14000\times X_b
  \]
\item
  Les contraintes sont les suivantes : \[
  \begin{align}
  20X_a + 40X_b &\leq 290 \\
  2X_a + 1X_b &\leq 1125 \\
  3X_a + 1X_b &\leq 120 \\
  X_a &\geq 0 \\
  X_b &\geq 0
  \end{align}
  \]
\end{itemize}

La première contrainte est le nombre de jour travaillé par an pour
l'utilisateur de notre outil. Il ne peut pas travailler plus de 290
jours par an.

La deuxième contrainte est le nombre de personnes à gérer par an. Il ne
peut pas gérer plus de 1125 personnes par an.

La troisième contrainte est le nombre d'intervenants à gérer par an. Il
ne peut pas gérer plus de 120 intervenants par an.

La quatrième et cinquième contrainte sont les contraintes de positivité
des variables de décision.

\n

Au niveau de la justification de nos contraintes :

\subsubsection{Nombre de jours travaillés par an
:}\label{nombre-de-jours-travailluxe9s-par-an}

Un salarié travaille environ 230 jours par an (on enlève les jours
fériés, 25 jours de congés, 52 samedis et 52 dimanche.

Partons du principe que notre wedding planner est auto-entrepreneur et
qu'il souhaite se lancer. Il veut également maximiser sa part de
bénéfices et il mettra ses jours fériés à profit pour travailler. En
effet, ses journées de travail ne sont pas très prenante puisqu'il
utilise une application qui permet d'optimiser toutes ses tâches.

Notre wedding planner travaillera alors environ 290 jours par an.

\subsubsection{Temps de préparation d'un mariage
:}\label{temps-de-pruxe9paration-dun-mariage}

\begin{itemize}
\tightlist
\item
  Un mariage de type A coûte moins cher puisqu'il y a moins de choses à
  préparer. Comme il y a moins préparatifs, le temps de préparation du
  mariage est 20 jours.
\item
  Un mariage de type B coûte plus cher qu'un mariage de type A,
  puisqu'il y a plus de préparatifs, plus de personnes a
  accueillir\ldots{} Ainsi, le temps de préparation d'un mariage de type
  B est 40 jours.
\end{itemize}

\subsubsection{Nombre de personnes à gérer par an
:}\label{nombre-de-personnes-uxe0-guxe9rer-par-an}

Un mariage de type A accueille 50 personnes. Un mariage de type B
accueille 120 personnes. Notre wedding planner ne peut pas gérer plus de
1125 personnes par an par soucis de stockage dans sa base de données.

\subsubsection{Nombre de logements à gérer par an
:}\label{nombre-de-logements-uxe0-guxe9rer-par-an}

Un mariage de type A nécessite 10 logements. Un mariage de type B
nécessite 22 logements. Notre wedding planner ne peut pas gérer plus de
10 logements par an par soucis d'organisation.

\begin{Shaded}
\begin{Highlighting}[]
\CommentTok{\# Fonction pour résoudre le modèle avec des coefficients légèrement perturbés}
\NormalTok{solve\_with\_perturbation }\OtherTok{\textless{}{-}} \ControlFlowTok{function}\NormalTok{(perturb\_factor) \{}
  \CommentTok{\# Perturber les coefficients des contraintes}
\NormalTok{  coef1 }\OtherTok{\textless{}{-}} \DecValTok{10} \SpecialCharTok{+} \FunctionTok{rnorm}\NormalTok{(}\DecValTok{1}\NormalTok{, }\DecValTok{0}\NormalTok{, perturb\_factor)}
\NormalTok{  coef2 }\OtherTok{\textless{}{-}} \DecValTok{20} \SpecialCharTok{+} \FunctionTok{rnorm}\NormalTok{(}\DecValTok{1}\NormalTok{, }\DecValTok{0}\NormalTok{, perturb\_factor)}
\NormalTok{  coef3 }\OtherTok{\textless{}{-}} \DecValTok{20} \SpecialCharTok{+} \FunctionTok{rnorm}\NormalTok{(}\DecValTok{1}\NormalTok{, }\DecValTok{0}\NormalTok{, perturb\_factor)}
\NormalTok{  coef4 }\OtherTok{\textless{}{-}} \DecValTok{10} \SpecialCharTok{+} \FunctionTok{rnorm}\NormalTok{(}\DecValTok{1}\NormalTok{, }\DecValTok{0}\NormalTok{, perturb\_factor)}
\NormalTok{  coef5 }\OtherTok{\textless{}{-}} \DecValTok{10} \SpecialCharTok{+} \FunctionTok{rnorm}\NormalTok{(}\DecValTok{1}\NormalTok{, }\DecValTok{0}\NormalTok{, perturb\_factor)}
\NormalTok{  coef6 }\OtherTok{\textless{}{-}} \DecValTok{15} \SpecialCharTok{+} \FunctionTok{rnorm}\NormalTok{(}\DecValTok{1}\NormalTok{, }\DecValTok{0}\NormalTok{, perturb\_factor)}

  \CommentTok{\# Résoudre le modèle avec les coefficients perturbés}
\NormalTok{  result }\OtherTok{\textless{}{-}} \FunctionTok{MIPModel}\NormalTok{() }\SpecialCharTok{|\textgreater{}}
    \FunctionTok{add\_variable}\NormalTok{(xA, }\AttributeTok{type =} \StringTok{"integer"}\NormalTok{, }\AttributeTok{lb =} \DecValTok{0}\NormalTok{) }\SpecialCharTok{|\textgreater{}}
    \FunctionTok{add\_variable}\NormalTok{(xB, }\AttributeTok{type =} \StringTok{"integer"}\NormalTok{, }\AttributeTok{lb =} \DecValTok{0}\NormalTok{) }\SpecialCharTok{|\textgreater{}}
    \FunctionTok{set\_objective}\NormalTok{(}\DecValTok{9500} \SpecialCharTok{*}\NormalTok{ xA }\SpecialCharTok{+} \DecValTok{14000} \SpecialCharTok{*}\NormalTok{ xB, }\StringTok{"max"}\NormalTok{) }\SpecialCharTok{|\textgreater{}}
    \FunctionTok{add\_constraint}\NormalTok{(coef1 }\SpecialCharTok{*}\NormalTok{ xA }\SpecialCharTok{+}\NormalTok{ coef2 }\SpecialCharTok{*}\NormalTok{ xB }\SpecialCharTok{\textless{}=} \DecValTok{290}\NormalTok{) }\SpecialCharTok{|\textgreater{}}
    \FunctionTok{add\_constraint}\NormalTok{(coef3 }\SpecialCharTok{*}\NormalTok{ xA }\SpecialCharTok{+}\NormalTok{ coef4 }\SpecialCharTok{*}\NormalTok{ xB }\SpecialCharTok{\textless{}=} \DecValTok{450}\NormalTok{) }\SpecialCharTok{|\textgreater{}}
    \FunctionTok{add\_constraint}\NormalTok{(coef5 }\SpecialCharTok{*}\NormalTok{ xA }\SpecialCharTok{+}\NormalTok{ coef6 }\SpecialCharTok{*}\NormalTok{ xB }\SpecialCharTok{\textless{}=} \DecValTok{350}\NormalTok{) }\SpecialCharTok{|\textgreater{}}
    \FunctionTok{solve\_model}\NormalTok{(}\FunctionTok{with\_ROI}\NormalTok{(}\AttributeTok{solver =} \StringTok{"glpk"}\NormalTok{))}

  \CommentTok{\# Extraire les solutions}
\NormalTok{  xA\_value }\OtherTok{\textless{}{-}} \FunctionTok{ifelse}\NormalTok{(}\SpecialCharTok{!}\FunctionTok{is.na}\NormalTok{(}\FunctionTok{get\_solution}\NormalTok{(result, xA)), }\FunctionTok{get\_solution}\NormalTok{(result, xA), }\ConstantTok{NA}\NormalTok{)}
\NormalTok{  xB\_value }\OtherTok{\textless{}{-}} \FunctionTok{ifelse}\NormalTok{(}\SpecialCharTok{!}\FunctionTok{is.na}\NormalTok{(}\FunctionTok{get\_solution}\NormalTok{(result, xB)), }\FunctionTok{get\_solution}\NormalTok{(result, xB), }\ConstantTok{NA}\NormalTok{)}
  
  \FunctionTok{return}\NormalTok{(}\FunctionTok{c}\NormalTok{(}\AttributeTok{xA =}\NormalTok{ xA\_value, }\AttributeTok{xB =}\NormalTok{ xB\_value))}
\NormalTok{\}}

\CommentTok{\# Nombre de simulations bootstrap}
\NormalTok{B }\OtherTok{\textless{}{-}} \DecValTok{1000}
\CommentTok{\# Facteur de perturbation (écart{-}type de la perturbation)}
\NormalTok{perturb\_factor }\OtherTok{\textless{}{-}} \DecValTok{1}

\CommentTok{\# Exécuter les simulations bootstrap}
\NormalTok{bootstrap\_results }\OtherTok{\textless{}{-}} \FunctionTok{replicate}\NormalTok{(B, }\FunctionTok{solve\_with\_perturbation}\NormalTok{(perturb\_factor))}

\CommentTok{\# Convertir les résultats en data frame pour analyse}
\NormalTok{bootstrap\_df }\OtherTok{\textless{}{-}} \FunctionTok{as.data.frame}\NormalTok{(}\FunctionTok{t}\NormalTok{(bootstrap\_results))}
\FunctionTok{colnames}\NormalTok{(bootstrap\_df) }\OtherTok{\textless{}{-}} \FunctionTok{c}\NormalTok{(}\StringTok{"xA"}\NormalTok{, }\StringTok{"xB"}\NormalTok{)}

\CommentTok{\# Afficher un résumé des solutions bootstrap}
\FunctionTok{summary}\NormalTok{(bootstrap\_df)}
\end{Highlighting}
\end{Shaded}

\begin{verbatim}
##        xA              xB        
##  Min.   :14.00   Min.   : 0.000  
##  1st Qu.:18.00   1st Qu.: 4.000  
##  Median :20.00   Median : 5.000  
##  Mean   :19.55   Mean   : 4.545  
##  3rd Qu.:21.00   3rd Qu.: 6.000  
##  Max.   :25.00   Max.   :10.000
\end{verbatim}

\begin{Shaded}
\begin{Highlighting}[]
\FunctionTok{library}\NormalTok{(ROI.plugin.glpk)}
\FunctionTok{library}\NormalTok{(ompr)}
\FunctionTok{library}\NormalTok{(ompr.roi)}

\NormalTok{result }\OtherTok{\textless{}{-}} \FunctionTok{MIPModel}\NormalTok{() }\SpecialCharTok{|\textgreater{}}
  \FunctionTok{add\_variable}\NormalTok{(xA, }\AttributeTok{type =} \StringTok{"integer"}\NormalTok{, }\AttributeTok{lb =} \DecValTok{0}\NormalTok{) }\SpecialCharTok{|\textgreater{}}
  \FunctionTok{add\_variable}\NormalTok{(xB, }\AttributeTok{type =} \StringTok{"integer"}\NormalTok{, }\AttributeTok{lb =} \DecValTok{0}\NormalTok{) }\SpecialCharTok{|\textgreater{}}
  \FunctionTok{set\_objective}\NormalTok{(}\DecValTok{9500} \SpecialCharTok{*}\NormalTok{ xA }\SpecialCharTok{+} \DecValTok{14000} \SpecialCharTok{*}\NormalTok{ xB, }\StringTok{"max"}\NormalTok{) }\SpecialCharTok{|\textgreater{}}
  \FunctionTok{add\_constraint}\NormalTok{(}\DecValTok{10} \SpecialCharTok{*}\NormalTok{ xA }\SpecialCharTok{+} \DecValTok{20} \SpecialCharTok{*}\NormalTok{ xB }\SpecialCharTok{\textless{}=} \DecValTok{290}\NormalTok{) }\SpecialCharTok{|\textgreater{}}
  \FunctionTok{add\_constraint}\NormalTok{(}\DecValTok{20} \SpecialCharTok{*}\NormalTok{ xA }\SpecialCharTok{+} \DecValTok{10} \SpecialCharTok{*}\NormalTok{ xB }\SpecialCharTok{\textless{}=} \DecValTok{450}\NormalTok{) }\SpecialCharTok{|\textgreater{}}
  \FunctionTok{add\_constraint}\NormalTok{(}\DecValTok{10} \SpecialCharTok{*}\NormalTok{ xA }\SpecialCharTok{+} \DecValTok{15} \SpecialCharTok{*}\NormalTok{ xB }\SpecialCharTok{\textless{}=} \DecValTok{350}\NormalTok{) }\SpecialCharTok{|\textgreater{}}
  \FunctionTok{solve\_model}\NormalTok{(}\FunctionTok{with\_ROI}\NormalTok{(}\AttributeTok{solver =} \StringTok{"glpk"}\NormalTok{))}

\FunctionTok{get\_solution}\NormalTok{(result, xA)}
\end{Highlighting}
\end{Shaded}

\begin{verbatim}
## xA 
## 19
\end{verbatim}

\begin{Shaded}
\begin{Highlighting}[]
\FunctionTok{get\_solution}\NormalTok{(result, xB)}
\end{Highlighting}
\end{Shaded}

\begin{verbatim}
## xB 
##  5
\end{verbatim}

\end{document}
